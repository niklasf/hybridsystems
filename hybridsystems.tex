\documentclass[ngerman]{beamer}
\usepackage{babel}
\usepackage[utf8]{inputenc}
\usepackage[T1]{fontenc}
\usepackage{lmodern}

\usetheme{TUC2}
\setbeamercovered{transparent}

\mode<presentation>
\title{Hybride Systeme}
\subtitle{Dependability and Trust}
\author{Daniel Arnsberger und Niklas Fiekas}
\institute{Institut für Informatik}
\date{04. Juli 2014}

\begin{document}

\begin{frame}
    \titlepage
\end{frame}

\begin{frame}
    \frametitle{Ein hybrides Szenario}
\end{frame}

\begin{frame}
    \frametitle{Sicherheit}
\end{frame}

\begin{frame}
    \frametitle{Stabilität}
\end{frame}

\begin{frame}
    \frametitle{Endliche Automaten}

    \begin{itemize}
        \item Erlauben einfaches Modelchecking
        \item Wie viele Zustände sind für das Szenario nötig?
    \end{itemize}
\end{frame}

\begin{frame}
    \frametitle{Hybride Automaten: Diskreter Teil}
\end{frame}

\begin{frame}
    \frametitle{Hybride Automaten: Kontinuierlicher Teil}
\end{frame}

\begin{frame}
    \frametitle{Ausblick}
\end{frame}

\end{document}
