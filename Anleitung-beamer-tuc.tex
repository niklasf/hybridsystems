
\documentclass[ngerman]{beamer}
\usepackage{babel}
\usepackage[latin1]{inputenc}
\usepackage[T1]{fontenc}
\usepackage{lmodern}

    \usetheme{TUC2}
%    \usefonttheme{structureitalicserif}
    \setbeamercovered{transparent}

\mode<presentation>
    \title{Folien im Corporate Design der TU~Clausthal}
    \subtitle{Die \LaTeX-Klasse beamer mit dem TUC-Theme}
    \author{D. W�sch}
    \institute{Institut f�r Mathematik}
    \date{19. Oktober 2005}


%%%%%%%%%%%%%%%%%%%%%%%%%%%%%%%%%%%%%%%%%%%%%%%%%%%%%%
\begin{document}
\begin{frame}
\titlepage
\end{frame}

\begin{frame}
\frametitle{Gliederung}
\tableofcontents
\end{frame}

%%%%%%%%%%%%%%%%%%%%%%%%%%%%%%%%%%%%%%%%%%%%%%%%%%%%%%
\section{Einf�hrung}

\begin{frame}
  \frametitle{Grunds�tzliches}
  \begin{itemize}
    \item Klasse: beamer
    \item Theme: TUC
    \item Syntax:
      \begin{semiverbatim}
\footnotesize
  \\documentclass[ngerman]\{beamer\}

  \\usepackage\{babel\}

  \\usepackage[latin1]\{inputenc\}

  \\usepackage[T1]\{fontenc\}

  \\usepackage\{lmodern\}

  \\usetheme\{TUC\}

  \\mode<presentation>

  \\begin\{document\}

  ...

  \\end\{document\}
      \end{semiverbatim}
      \item Zudem interessant im Vorspann:
      \begin{semiverbatim}
        \\setbeamercovered\{transparent\}
      \end{semiverbatim}
      Bewirkt, dass versteckte Punkte `d�nn' sichtbar sind
    \end{itemize}
\end{frame}

\begin{frame}
  \frametitle{Aufbau einer Folie}
  \begin{itemize}
    \item Umgebung f�r Folie:
    \begin{semiverbatim}
      \\begin\{frame\}\dots\\end\{frame\}
    \end{semiverbatim}
    \item Au�erhalb kann mit section-Befehlen eine Struktur angegeben werden, die als Inhaltsverzeichnis oder Vortragsstruktur ausgegeben werden kann
    \item Der Folientitel wird mit dem Befehl
    \begin{semiverbatim}
      \\frametitle\{\textrm{�berschrift}\}
    \end{semiverbatim}
    festgelegt.
  \end{itemize}
\end{frame}

%%%%%%%%%%%%%%%%%%%%%%%%%%%%%%%%%%%%%%%%%%%%%%%%%%%%
\section{Nummerierungen und Aufz�hlungen}
\begin{frame}
  \frametitle{Gliederungs-Befehle}
  \begin{itemize}
    \item die �blichen Umgebungen wie \texttt{enumerate} und
        \texttt{itemize} funktionieren wie gewohnt und sind dem Corporate
        Design angepasst
    \begin{itemize}
      \item Auch die Unter-
      \begin{itemize}
        \item und die Unterunteraufz�hlungszeichen
      \end{itemize}
      \item sind angepasst.
    \end{itemize}
  \end{itemize}
  \begin{enumerate}
    \item Hier
    \item ist
    \begin{enumerate}
      \item eine
      \item nummerierte
      \begin{enumerate}
        \item Aufz�hlung.
      \end{enumerate}
    \end{enumerate}
  \end{enumerate}
\end{frame}

\begin{frame}
  \frametitle{Beschreibungen}
  \begin{description}
    \item[Beschreibungen] funktionieren auch wie �blich mit der \texttt{description}-Umgebung.
    \item[Stichw�rter] sind hierbei im CD-gr�n gehalten. Ist die Beschreibung l�nger, so wird diese umgebrochen und links einger�ckt.
  \end{description}

  \begin{itemize}
    \item Um \alert{einzelne W�rter} im Text hervorzuheben, dient der Befehl
    \begin{semiverbatim}
      \\alert\{\textrm{Text}\}
    \end{semiverbatim}
    \item F�r diese Hervorhebung wird die zweite Zierfarbe des Corporate Design benutzt.
  \end{itemize}
\end{frame}

%%%%%%%%%%%%%%%%%%%%%%%%%%%%%%%%%%%%%%%%%%%%%%%%%%%%%%
\section{Overlays}
\begin{frame}
  \frametitle{Von oben nach unten}
  \pause
  \begin{itemize}
    \item Mit dem Befehl
    \begin{semiverbatim}
      \\pause
    \end{semiverbatim}
    k�nnen Folien von oben nach unten aufgebaut werden.
    \pause
    \item Dabei wird f�r jedes ``pause'' eine neue Folie angelegt.
    \pause
    \item Die Nummerierung rechts unten erfolgt aber nach der Frame-Anzahl, d.\,h. diese ist innerhalb eines Frames konstant und entspricht nicht der tats�chlichen Seitenzahl des Dokumentes.
    \pause
    \item Einen �berblick �ber die gesamten M�glichkeiten (z.\,B. auch Befehle wie \texttt{visible}, \texttt{uncover}, \texttt{only}, \dots) gibt die Datei \texttt{beameruserguide.pdf}.
  \end{itemize}
\end{frame}

\begin{frame}
  \frametitle{Beliebige Reihenfolgen}
    �ber
    \begin{semiverbatim}
      \\item<\textrm{$n$--$m$}> \textrm{bla}
    \end{semiverbatim}
    kann eine beliebige Reihenfolge vorgegeben werden.

    \begin{Beispiel}
    Der folgende Frame hat in der \texttt{itemize}-Umgebung folgenden Code:
    \begin{semiverbatim}
    \\item<2-> ab zweitens

    \\item<1-2> erstens und zweitens

    \\item<2> nur zweitens

    \\item<-3> bis drittens

    \\item<4> viertens
    \end{semiverbatim}
    \end{Beispiel}
\end{frame}

\begin{frame}
  \frametitle{Overlay-Beispiel}
  \begin{itemize}
    \item<2-> ab zweitens
    \item<1-2> erstens und zweitens
    \item<2> nur zweitens
    \item<-3> bis drittens
    \item<4> viertens
  \end{itemize}
\end{frame}

\begin{frame}
  \frametitle{Farben}
\sloppy
\begin{itemize}
\item Die CD-Farben der Universit�t hei�en:
\begin{description}\itemsep0pt
  \item[\textcolor{TUCgreen}{TUCgreen}]     das TUC-Gr�n (im S/W: schwarz)


  \item[\textcolor{TUCgrey1}{TUCgrey1}]     50 \% grau (Dunkelgrau, im S/W: schwarz)


  \item[\textcolor{TUCgrey2}{TUCgrey2}]     10 \% grau (Hellgrau, im S/W: wei�)


  \item[\textcolor{TUCred}{TUCred}]       die rote Zierfarbe (im S/W: schwarz)


\end{description}
\item \begin{semiverbatim}
\rm
Die Farben k�nnen mit \texttt{\\color\{farbname\}} bzw.  \texttt{\\textcolor\{farbname\}\{text\}} verwendet werden.
\end{semiverbatim}
\end{itemize}
\end{frame}

%%%%%%%%%%%%%%%%%%%%%%%%%%%%%%%%%%%%%%%%%%%%%%%%%%%%%%
\section{Mathematisches}
\begin{frame}
  \frametitle{Mathematische Umgebungen}
  \begin{Fakt}
    Es existieren die Umgebungen
    \texttt{theorem},
    \texttt{proof},
    \texttt{corollary},
    \texttt{fact},
    \texttt{lemma},
    \texttt{problem},
    \texttt{solution},
    \texttt{definition},
    \texttt{example},
    \texttt{definitions} und
    \texttt{examples}, sowie
    \texttt{Satz},
    \texttt{Beweis},
    \texttt{Folgerung},
    \texttt{Fakt},
    \texttt{Lemma},
    \texttt{Problem},
    \texttt{Loesung} und
    \texttt{Definition}.
  \end{Fakt}
\end{frame}

\begin{frame}
  \frametitle{Formeln auf Folien}
  \begin{Satz}
    Umgebungen wie \$\$ \dots \$\$, \texttt{align} usw. funktionieren wie gewohnt.
  \end{Satz}
  \begin{Beweis}
    Hier ist eine Formel: $e^{i\pi{}} + 1 = 0$.
    Jetzt kommen mehrere:
    \begin{align*}
      \int \sin ax\ dx &= - \frac1a \cos ax \\
      \int \sin^2 ax\ dx &= \frac12x - \frac1{4a} \sin 2ax
    \end{align*}
  \end{Beweis}
\end{frame}

%%%%%%%%%%%%%%%%%%%%%%%%%%%%%%%%%%%%%%%%%%%%%%%%%%%%%%
\section{Fazit}
\begin{frame}
  \frametitle{Fazit}
  \begin{itemize}
    \item �ber die \texttt{beamer}-Klasse stehen m�chtige Folien-Befehle zur Verf�gung
    \item Mit dem Theme \texttt{TUC} ist das Corporate Design der TU Clausthal umgesetzt
    \item Insgesamt lassen sich damit Folien gestalten, die allerdings keine Navigationselemente enthalten, da das Design diese nicht vorsieht
    \item Durch das eingebundene \texttt{hyperref}-Paket lassen sich aber dennoch Links innerhalb des Dokumentes problemlos realisieren
  \end{itemize}
\end{frame}

\end{document}
